\documentclass{article}
\usepackage[polish]{babel}
\usepackage[utf8]{inputenc}
\usepackage{polski}
\frenchspacing
\usepackage{indentfirst}
\usepackage{amsmath}
\title{Sprawozdanie z numerycznych metod rozwiązywania cząstkowych równań rózniczkowych}
\author{Karol Urbański}
\begin{document}
\maketitle
\section{Podstawy teoretyczne}
\subsection{Rozwiązywany problem}
Rozwiązywany problem to zadanie 1 z laboratorium, w którym znajdujemy rozwiązanie \emph{równania Laplace'a} 
w przestrzeni dwuwymiarowej $(x,y)\in [0,1]^2$:
	\begin{equation}
		\frac{\delta^2 V}{\delta x^2}
			+
		\frac{\delta^2 V}{\delta y^2}
			=
		0
	\end{equation}
z następującymi warunkami początkowymi:
	\begin{equation*}
		\forall x,y \in R: V(x,0)=V(x,1)=V(0,y)=V(1,y)=0,
	\end{equation*}
	\begin{equation*}
		\forall (x,y) \in C: V(x,y)=l,
	\end{equation*}
gdzie $C$ to kwadratowy obszar przewodnika taki, że $C \in [0,1]^2$,
a $l$ to dana stała oznaczająca potencjał przewodnika.
\subsection{Dyskretyzacja problemu}
Przestrzeń, w której rozwiązujemy równanie, musimy udyskretyzować, aby móc wykorzystać metody
numeryczne do przybliżenia rozwiązania problemu. Dyskretyzacji dokonujemy poprzez podział przestrzeni
na siatkę $(n-1)^2$ punktów $p_{ij} \in [0,1]^2$ takich, że $\forall (i,j) \in \{1,2,...,n-1\}\times \{1,2,...,n-1\}: p_{ij}=(ih,jh), h=\frac{1}{n}$.
\newpage
\begin{figure}[h]
\begin{center}
\setlength{\unitlength}{1cm}
\begin{picture}(4,4)
	\linethickness{0.075mm}
	\multiput(0,0)(4,0){2}%
		{\line(0,1){4}}
	\multiput(0,0)(0,4){2}%
		{\line(1,0){4}}
	\put(1,1){\circle*{0.05}}
	\put(1,2){\circle*{0.05}}
	\put(1,3){\circle*{0.05}}
	\put(2,3){\circle*{0.05}}
	\put(3,3){\circle*{0.05}}
	\put(3,2){\circle*{0.05}}
	\put(3,1){\circle*{0.05}}
	\put(2,1){\circle*{0.05}}
	\put(2,2){\circle*{0.2}}

\end{picture}
\parbox{4in}{\emph{\caption{Przykładowa dyskretyzacja dla $n=4$, punkty oznaczone kropkami, punkty należące do przewodnika w centrum (pogrubiony); obwódka oznacza otaczający ekran}}}
\end{center}
\end{figure}

Ponieważ do rozwiązania problemu będziemy
wykorzystywać MRS\footnote{MRS - metoda różnic skończonych} dla równania stacjonarnego, dla
ułatwienia obliczeń musimy zastosować dyskretyzację w taki sposób, by zmapować siatkę dwuwymiarową
do  jednowymiarowej. Wykonujemy ją w najprostszy możliwy sposób: \[p_k=p_{ij} \iff k=i\cdot (n-1)+j \]
czyli poprzez przypisanie kolejnych indeksów kolejnym elementom w kolejnych wierszach (idąc od góry 
do dołu oraz od lewej do prawej). Przypisanie to jest bardzo proste, ale niezbędne jest zadbanie o
 dokładne przygotowanie obliczeń, aby uniknąć błędów.
\end{document}
